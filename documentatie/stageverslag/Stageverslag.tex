\documentclass{article}
\usepackage{graphicx} % Required for inserting images
\usepackage{todonotes}
\usepackage{hyperref}
\usepackage{listings}
\usepackage{rotating}
\usepackage{blindtext}
\usepackage[a4paper, total={6in, 8in}]{geometry}

\title{Stage Verslag 2025 - 2026}
\author{Olaf Goudriaan 1071349}
\date{TINSTG05 - Wessel Oele, Renee van Dorn}
\begin{document}
\sffamily
	\begin{titlepage}
		\begin{center}
			\includegraphics[width=8cm]{Sens2SeaLogo.png}

			\vspace*{4cm}
			\Huge
			\textbf{Aansturing freesbank} \\
			\LARGE
			\vspace{0.5cm}
        Stage verslag \\ 
				Sens2Sea
			
			\vspace{1.5cm}
					
			\huge
			\textbf{Olaf Goudriaan}
			\vspace{1.5cm}	
			\vfill
			\Large
			\begin{flushright}
				Geert Mosterdijk \\
				TINSTG05 \\
				Wessel Oele, Renee van Dorn \\
				1071349	\\
				2025 - 2026
			\end{flushright}
			
		\end{center}

	\end{titlepage}
	\newpage
 	\renewcommand{\familydefault}{\sfdefault}
	\tableofcontents
	\newpage
	\large

	\paragraph{Voorwoord}
	Ik dank hierbij iedereen die heeft geholpen met de stage. Ik ben veel gegroeid in dit half jaar in hard en soft skills.


	\section{Context}
		\subsection{Het bedrijf}
			De stage heeft plaatsgevonden bij het bedrijf Sens2Sea. Sens2Sea is een klein bedrijf dat gevestigd is op het RDM terrein. Het bedrijf heeft een kavel in het Innovation Dock. Sens2Sea ontwikkelt radar technologieën voor sectoren zoals de scheepsvaart en andere maritieme sectoren. Het bedrijf wordt geleid door Geert Mosterdijk. Het bedrijf heeft maar een paar medewerkers. Deze zijn gespecialiseerd in hun vakgebieden zoals: elektronica en software. Verder lopen er bij Sens2Sea een aantal studenten stage van verschillende opleidingen. Het bedrijf werkt voor projecten samen met andere bedrijven. Deze bedrijven zijn tevens de stakeholders voor de projecten. Voor dit bedrijf geldt het volgende:
			\begin{enumerate}
				\item Sens2Sea. De opdracht is voor Sens2Sea, dit bedrijf bepaalt welke beslissingen gemaakt worden. Dit bedrijf heeft mensen in dienst die expertise bieden op gebied van elektronica en software. (Hoge invloed en Hoog belang)
				\item PKMarine. Dit bedrijf werkt samen met Sens2Sea en bied expertise in regeltechniek. (Hoge invloed, laag belang)
				\item CoE HRTech. Dit bedrijf financiert het project. (Lage invloed en hoog belang)
			\end{enumerate}

		\subsection{Opdracht}
			\paragraph{} Sens2Sea is bezig met radar technologie. Radar is een technologie die radiogolven gebruikt om, t.o.v. de positie van de radar, de afstand, snelheid en richting van objecten te bepalen. Radar kan gebruikt worden in goed en slecht weer. Hierdoor is het uitstekend geschikt om objecten te detecteren in sectoren zoals de scheepsvaart. Ook kan radar technologie gebruikt worden om de zeebodem in kaart te brengen en om olievlekken te detecteren. 

			\paragraph{} Sens2Sea is sinds kort bezig FMCW (Frequency Modulated Constant Wave) radar technologie. Hiervoor wilt Sens2Sea parabool antennes gebruiken. Deze antennes zorgen ervoor dat de radiogolven gebundeld worden. Door de bundeling van deze golven worden de signalen versterkt. Dit zorgt ervoor dat het beeld een betere resolutie heeft. Deze antennes zijn niet te verkrijgen in alle maten. Deze zouden custom op maat gemaakt moeten worden. Dit is ten opzichte van standaard geproduceerde maten een stuk duurder. Machines die al op de markt zijn en met een werkgrote van 4m x 1m x 0.4m, groot genoeg zijn kosten tienduizenden euro's. Daarom wil Sens2Sea eigen antennes maken met een eigen 3D freesbank. Deze opdracht omvat het realiseren van een nauwkeurige aansturing van een 3D freesbank.
			
		\subsection{Scope} 
			\paragraph{} Om een freesbank aan te sturen zijn er al opensource projecten zoals GRBL uitgevoerd. Deze projecten voldoen helaas niet aan de eisen omdat Sens2Sea wil dat de software en hardware makkelijk aanpasbaar is. Bovendien moet de freesbank over een afstand van vier meter nauwkeurig kunnen werken. Hiervoor is het van belang de aandrijving van de frees en de positie bepaling van de snij kop gescheiden te houden. Dit is niet het geval met deze opensource software zoals GRBL. Bij deze oplossingen bepaalt het systeem de positie van de kop door de stappen te tellen die de motoren moeten uitvoeren. Een probleem hiervan is dat de motoren een stap kunnen verspringen waardoor de gewenste kwaliteit van de vorm van de antenne niet kan worden gegarandeerd. Daarom wil Sens2Sea dat de positie van de snij kop door sensoren wordt bepaald opdat deze kop nauwkeurig kan worden gepositioneerd in zowel de als de X, Y en Z richtingen. 
			
			\paragraph{} De scope van het project omvat het scheiden van de aandrijving en de positie bepaling van de snij kop. Daarnaast is er sprake van aanpassen of herschrijven van GRBL om deze te laten werken met de sensoren. Ook de hardware aansturen/aansluiten hoort bij de scope.

		\subsection{Mogelijke oplossingen}
			Een mogelijke oplossing voor het probleem zouden stepper motor met encoder kunnen zijn. Dit zou handig zijn omdat de code van GRBL niet aangepast hoeft te worden. Deze encoders zitten op de achterkant van de encoder en meten of de encoder wel de stap heeft gezet die die moest zetten. Indien hier een fout in zit zal de encoder compenseren.

			Dit is voorgelegd aan de opdrachtgever. Maar dat is geen oplossing omdat afhankelijkheid van de fabrikant wordt gemaakt. Dit omdat de fabrikant de encoder bepaald op de motor en hierdoor de nauwkeurigheid van de encoder bepaald. Sens2Sea wilt een systeem dat aan te passen is naar de wensen die op dat moment belangrijk zijn. En indien nodig andere sensoren toevoegen/gebruiken voor de bepaling van de positie van de snij kop. 



	\newpage
	\section{Analyse}
		\subsection{Huidige situatie}
			\begin{figure}
				\begin{center}
				\includegraphics[width=10cm]{beginsysteem.png}
				\caption{systeem diagram}
				\end{center}
			\end{figure}
			De huidige situatie bestaat uit een aantal functies die te zien zijn in het figuur ... \todo{figuur}. De functies die al aanwezig zijn, zijn de motor drivers, actuatoren, freesbank met frees en een regelsysteem. 
			

			\todo{frees valt buiten scope, microstepping buiten scope, verwijzing datasheet?}
			\subsubsection{Huidige functies}
				\paragraph{\textbf{Freesbank}}Het frame is gemaakt van hout. De werkruimte van het frame bedraagt 300cm in de lengte (X-as), 142,5cm in breedte (Z-as) en 8 cm in hoogte (Y-as). De werkelijke lengte van de freesbank is 251cm omdat de wagen op de rails aan beide kanten uitsteekt. Aan de zijkant is een verhoging dat 5cm breed is aan beide kanten. Hierop zijn geleiderails gemonteerd waarover de wagen beweegt. De effectieve werkbreedte is $142.5 - 8 - 7=127.5$cm omdat de frees niet helemaal tot de zijkant kan komen. Het hoogste punt van de wagen is 60cm boven het bed van de freesbank. Op deze wagen is een tweede railsysteem bevestigd voor de beweging in de breedte-as. Hierop bevind zich een tweede wagen voor de hoogte-as. De hoogte beweging wordt gerealiseerd door middel van een draaiende schroef, waardoor de snij kop gecontroleerd omhoog en omlaag kan bewegen. Aan deze wagen is een frees gemonteerd. De aandrijving van de lengte- en breedte-as gebeurt met tandriemen. Deze lopen over de volledige lengte van de assen en worden aangedreven door timing pulleys. Dit zijn tandwielen voor tandriemen. \\
				{\color{white}-}\\
				Op de wagen van de hoogte-as is een handfrees gemonteerd. Deze frees werkt op netstroom. Als de frees snijt, komt er materiaal los. Dit hoopt op op de freesbank. De ophoping en frees aansturen met de microcontroller of een nieuwe frees zoeken is buiten de scope van het project.\\
				{\color{white}-}\\
				De freesbank heeft externe invloeden zoals vrijving op de rails en uitrekking van de tandriemen.

				\paragraph{Actuatoren}De motoren die worden gebruikt voor de beweging van de assen zijn NEMA 17 motoren. Deze werken op 12-36V en mogen maximaal 2A hebben. De motoren zitten vast aan het frame met bevestigings punten. Aan de motoren zit een timing pulley bevestigd, waar een tandriem overheen loopt. De lengte as heeft motoren aan beide kanten van de freesbank. De hoogte en breedte assen worden aangedreven door één motor.

				\paragraph{Drivers}De drivers die worden gebruikt voor de motoren zijn DRV8824 drivers. Deze drivers werken op 8.2-45V en mogen maximaal 2.5A hebben op 24V. Deze drivers maken gebruik van een STEP en DIR pin. Door de DIR pin hoog en laag te maken veranderd de polariteit over de aangesloten motor. Door de STEP pin hoog te maken zet de motor een stap. Deze drivers hebben ook de mogelijkheid om te microsteppen. Microsteppen maakt gebruik van sinusvormige stromen om meer nauwkeurigheid en hogere resolutie te krijgen. De drivers krijgen voeding van een labvoeding en het regelsysteem.

				\paragraph{Regelsysteem} Het systeem wordt geregeld doormiddel van een arduino uno met een opensource regelsysteem GRBL. Dit is het onderdeel dat input van de gebruiker krijgt en informatie terug stuurd naar de gebruiker.

			\subsubsection{Oplossingen}
				In interviews met de opdrachtgever is voorgelegd of een steppermotor met ingebouwde encoders een opslossing zou zijn. Hierdoor zou de code van GRBL niet aangepast hoeven worden. Deze encoders zijn closed loop en controlleren of de stappen motor de stap zet die die moet zetten en de fout kan herstellen. De opdrachtgever gaf aan dat dit niet de gewenste oplossing is, omdat er dan een afhankelijkheid van de fabrikant word gemaakt. Dit omdat de fabrikant bepaald welke encoder op de stappenmotor komt, waardoor de fabrikant bepaald hoe nauwkeurig de freesbank kan zijn. Ook wilt Sens2Sea het systeem gemakkelijk kunnen aanpassen wanneer er meer nauwkeurigheid gewenst is.

			\subsubsection{Nieuwe functies}
				\begin{figure}
					\begin{center}
					\includegraphics[width=10cm]{beginsysteem.png}
					\caption{Nieuw systeem diagram}
					\end{center}
				\end{figure}

				\paragraph{Sensor}Uit een interview met de opdrachtgever is gebleken dat de opdrachtgever encoders wilt gebruiken en wilt bevestigen op een getande riem die op het frame is geplakt. Hierdoor kan de getande riem niet uitrekken. De nauwkeurigheid van de encoder is niet het belangrijkste omdat het project gaat over het scheiden van de aandrijving van de frees en positie bepaling van de snij kop. Verder doen de kosten er toe, maximaal 60 euro per encoder. Ook is er belang bij dat de huidige positie opvragen snel kan aangezien een freesbank een high performance machine is. Er zijn verschillende soorten encoders die verschillend werken. De nauwkeurigheid van een encoder wordt gemeten in Pulses Per Rotation(PPR). Dit geeft aan dat hoe hoger de PPR is de hoger de nauwkeurigheid omdat er meer pulsen gegeven worden in de zelfde rotatie. Na het onderzoek zijn de opties voorgelegd aan de opdrachtgever en is er gekozen om een optische incrementele encoder te gebruiken. Er zijn ook externe invloeden op de encoder zoals vuil dat in de encoder komt, dat de metingen kan verstoren.
				
				\paragraph{Regelaar}Als regelaar is een microcontroller nodig. De keuze van de encoder zorgt er voor dat de microcontroller 6 interrupt pinnen nodig heeft. Verder moet deze rond de specificaties zijn als een arduino uno. Dit is omdat opensource oplossingen zoals GRBL gebaseerd zijn op een arduino uno. Hierom is voorgesteld om een arduino mega te gebruiken. Deze mircrocontroller heeft voldoende rekenkracht voor dit soort applicaties. Hiernaast heeft deze microcontroller ook de benodigde 6 interrupt pinnen voor de drie verschillende assen. Daarnaast bieden de vele pinnen mogelijkheden om het systeem uit te breiden.
			
			\subsubsection{Software}
				\begin{figure}
					\begin{center}
					\includegraphics[width=10cm]{microcontrollersysteem.png}
					\caption{microcontroller systeem}
					\end{center}
				\end{figure}
				De software op de microcontroller bestaat uit verschillende onderdelen zoals te zien is in het figuur. Hoe de voeding binnen de microcontroller loopt is weggelaten. De datastromen zijn ook te zien in het figuur. \\
				{\color{white}-}\\
				\textbf{Actuatoren aansturing} is het onderdeel wat de actuatoren aanstuurt. Dit onderdeel beïnvloed de pinnen om de motor driver aan te sturen. \\
				{\color{white}-}\\
				\textbf{Sensoren uitlezer} is het onderdeel dat de sensoren uitleest. Hierna zet de sensor de data om naar bruikbare positie data. \\
				{\color{white}-}\\
				\textbf{Gebruiker input en output} is het onderdeel dat de data ontvangt en verstuurd van de gebruiker. \\
				{\color{white}-}\\
				\textbf{Gebruiker input verwerker} is het onderdeel dat de input van de gebruiker verwerkt en omzet naar bruikbare opdrachten. Ook zend dit onderdeel opdrachten naar de regelaars. Ook handelt dit onderdeel alle andere opdrachten voor het systeem af. Dit zijn opdrachten zoals status opvragen en terug sturen, andere opdrachten voor de bewegingen die niet afgehandeld door de regelaars. \\
				{\color{white}-}\\
				\textbf{De regelaars} zijn de onderdelen die per as bepalen hoe de actuatoren bewogen moeten worden. Dit doen ze door de positie data van de sensor uitlezer te vergelijken met de positie waar de actuator moet zijn. \\

		\subsection{Requirements}
			Uit de analyse en aan de hand van meerdere interviews met de opdrachtgever zijn de volgende requirements opgekomen.
			\begin{center}
				\scalebox{0.7}{\begin{tabular}{|c|c|c|c|}
					\hline
					Requirement & User story & Acceptatie & soort \\
					\hline
					De snij kop van de machine moet over& Als opdrachtgever wil ik dat de machine & Als de kop zich over de 	& Funct.	\\
					de X, Y en Z as kunnen bewegen. 		&	zich over de X, Y en Z as kan bewegen. 	& X, Y en Z as kan bewegen.	& 				\\
					\hline
					De positie van de snij kop moet  		& Als opdrachtgever wil ik dat de machine	& Als de positie van de			& Funct. 	\\
					bepaald worden doormiddel van   		& de positie van de kop bepaald doormiddel& snij kop bepaald word  		&				 	\\
					sensoren.														& van sensoren on de nauwkeurigheid te 		& door een sensor.					&				 	\\
																							& garanderen.															&														&					\\
					\hline
					Er kan g-code uitgevoerd worden			& Als opdrachtgever wil ik een g-code  		& Als de machine g-code kan & Funct. 	\\
					door de freesbank. 									& bestand laten uitvoeren door de 				& uitvoeren.								&					\\
																							& freesbank.															&														&					\\
					\hline
					De code van het prototype moet 			& Als opdrachtgever wil ik gemakkelijk 		& Als er in de code van de 	& Niet		\\
					modulair zijn waardoor gemakkelijk	& onderdelenkunnen vervangen en toevoegen & machine componenten				& Funct.	\\
					delen vervangen kunnen worden,			& aan de machine, zonder dat de hele code & toegevoegd, vervangen of	&					\\
				  zoals actuatoren en sensoren.				&	omgebouwd moet worden.									& verwijderd kunnen worden.	&					\\
					\hline
				\end{tabular} }
			\end{center}
	
	\section{Ontwerpen}

		\subsection{Software ontwerpen}
			Voor de software zijn verschillende ontwerpen gemaakt. Hieronder vallen de state machine, software flowchart en sequentie diagram.

			\paragraph{State machine} In software diagram A is de state machine te zien. Hierin is te zien in welke states het systeem zich kan bevinden. Het systeem start bij 'Idle'. Hierna kan het systeem opdrachten krijgen waardoor de state 'Running' word. De opdrachten kunnen klaar zijn of worden geannuleerd. Hierna wordt de state veranderd naar 'Idle'. Er kunnen fouten optreden, Als dit gebeurt zijn er 2 states. Indien het systeem fouten vind dat deze zelf kan verhelpen veranderd de state naar 'Out of service'. Indien de fouten niet te verhelpen zijn moet er een restart komen door een mens en zal de state 'END PROGRAM RESTART REQUIRED' worden. Uit deze state kan niet terug gekomen worden door de machine dus is er een restart nodig.
			\begin{center}
				\includegraphics[width=10cm]{statemachine.png}\\
				Software diagram A: State machine.
			\end{center}

			\paragraph{Flowchart} In software diagram B is een flowchart te zien. De flowchart is verdeeld in meerdere kleuren. Oranje en geel zijn interrupts voor de input van de gebruiker. Rood is de interrupt van de senzoren en paars de interrupt die de motoren aanstuurt. Het blauwe gedeelte is de main loop die input, errors en het maken van opdrachten afhandeld. Als er een input van de gebruiker komt, wordt die gelezen. Indien het een systeem command is zoals de status opvragen zal die meteen worden afgehandeld door 'execute system command'. Anders word de G-code gelezen en in de rij queue gezet. Als het G-code commando meteen afgehandeld moet worden, zal dat gedaan worden.
			\begin{center}
				\includegraphics[width=8cm]{Flowchart-full.png}\\
				Software diagram B: Flowchart.
			\end{center}

			\paragraph{sequentie diagram}

		\subsection{Hardware ontwerpen}
		\paragraph{Elektrische schema} Voor het elektrisch schema zijn twee schema's gemaakt die te vinden zijn in appendix B. \todo{controleren appendix nummer} Het eerste ontwerp is gemaakt voor de aansturing op een as. Dit schema bevat een arduino uno, encoder, DRV8824 motor driver en een NEMA17 stepper motor. Het tweede elektrische schema is het schema voor de drie assen. Dit schema bevat een arduino mega, 3x encoder, 3x DRV8824 motor driver, 3x NEMA17 stepper motor.

		\subsection{Encoder bevestigings punt Z-as}
			De encoder moet lopen over een randriem die vast zit op de Z-as. Hiervoor is een beugel nodig  



		\todo{bevestiging encoder iteraties, }
			

	\section{Realisatie}






	\section{Test Rapport}
		\todo{Work in progress}
		In dit plan worden de eisen/requirements van het prototype getest om te kijken of ze behaald zijn.

		\subsection{De kop van de machine moet over drie assen bewegen}
			\subsubsection{Acceptatiecriteria:}
				Deze test is voltooid als de kop van de machine kan bewegen over de drie bewegingsassen (x, y, z).

			\subsubsection{Testopstelling}
				Deze test kan alleen op de machine worden uitgevoerd. Benodigd is: 
				\begin{enumerate}
					\item Testopstelling.
					\item Handleiding \todo{work in progress}
				\end{enumerate}

			\subsubsection{Instructies}
				\begin{enumerate}
					\item Volg de handleiding om de machine aan te zetten.
					\item Upload een simpel stuk gcode om te machine over alle drie de assen te laten bewegen.
				\end{enumerate}

			\subsubsection{Verwacht resultaat}
				Het verwachte resultaat is dat de kop van de machine beweegt naar het punt uit de instructies.

			\subsubsection{Waarnemingen}
				\todo{waarnemingen invullen}
				

	\section{Aanbevelingen}
		Bij deze opdracht zijn meerdere onderdelen aan bod gekomen waar dit project niet aan toe gekomen is.
		\begin{enumerate}
			\item Er zijn modules waarmee de encoders makkelijker uit te lezen zijn. Hier moet onderzoek naar gedaan worden.
			\item rotzooi afzuigen frees
			\item 
		\end{enumerate}
		\todo{aanbevelingen toeveogen}
	\section{Bronnen}
		\begin{enumerate}
			\item Wikipedia contributors, “Radar,” Wikipedia, The Free Encyclopedia (Dutch version). [Online]. Available: \url{https://nl.wikipedia.org/wiki/Radar}. Accessed: Jan. 6, 2026.
			\item Wikipedia contributors, “Radar,” Wikipedia, The Free Encyclopedia. [Online]. Available: \url{https://en.wikipedia.org/wiki/Radar}. Accessed: Jan. 6, 2026.
			\item Wikipedia contributors, “Incremental encoder,” Wikipedia, The Free Encyclopedia. [Online]. Available: \url{https://en.wikipedia.org/wiki/Incremental_encoder}. Accessed: Jan. 6, 2026.
			\item Wikipedia contributors, “Rotary encoder,” Wikipedia, The Free Encyclopedia. [Online]. Available: \url{https://en.wikipedia.org/wiki/Rotary_encoder}. Accessed: Jan. 6, 2026.
			\item Laumans Techniek, “Encoders voor industriële automatisering,” Laumans Techniek. [Online]. Available: \url{https://www.laumanstechniek.nl/blog/kennisbank-3/encoders-voor-industriele-automatisering-20}. Accessed: Jan. 6, 2026.
			\item PBC Linear, “Stepper Motor Support: NEMA 17 Data Sheet,” PBC Linear. [Online]. Available: \url{https://media.pbclinear.com/pdfs/pbc-linear-data-sheets/data-sheet-stepper-motor-support.pdf}. Accessed: Jan. 6, 2026.
			\item Texas Instruments, “DRV8824 Stepper Motor Controller IC,” Datasheet. [Online]. Available: \url{https://www.ti.com/lit/ds/symlink/drv8824.pdf}. Accessed: Jan. 6, 2026.
			\item Linear Motion Tips, “Microstepping Basics,” Linear Motion Tips. [Online]. Available: \url{https://www.linearmotiontips.com/microstepping-basics/}. Accessed: Jan. 6, 2026.
			\item Dynapar, “Magnetic Encoders,” Dynapar Knowledge Center. [Online]. Available: \url{https://www.dynapar.com/knowledge/encoder-basics/encoder-technology/magnetic-encoders}. Accessed: Jan. 8, 2026.
			\item Wikipedia contributors, “Gray code,” Wikipedia, The Free Encyclopedia. [Online]. Available: \url{https://en.wikipedia.org/wiki/Gray_code}. Accessed: Jan. 8, 2026. 
		\end{enumerate}


		\appendix
		
			\section{Encoder onderzoek}
				Voor de positie van de snij kop van een freesbank te bepalen zijn verschillende mogelijkheden. Uit een interview met de opdrachtgever is gekomen dat deze encoders wilt gebruiken. Een encoder is een sensor met een roterende as. Als deze as draait zet de encoder de beweging om in elektrische signalen. Hierdoor kunnen microcontrollers de positie, snelheid en richting bepalen. De encoders zijn onder te verdelen in verschillende soorten: Incrementeel of absoluut. Ook zijn er verschillende meet methodes: Magnetisch of optisch. Er zijn ook encoders die lineair zijn, maar daarvan heeft de opdrachtgever aangegeven dat die niet bruikbaar zijn. Bij de keuze moet ook rekening gehouden worden met dat de prijs van een sensor onder de 60 euro moet blijven. 

				\subsection{Meet methodes}
					Er zijn verschillende manieren om de beweging te meten. Hiervoor zijn optisch en magnetisch de meest gebruikte technologieën.

					\textbf{Magnetisch} Magnetische encoders maken gebruik van een magnetisch veld om de rotatie van de as te bepalen. Dit heeft het voordeel dat deze robuuster zijn. Hiernaast zijn deze ook beter bestand tegen vervuiling, trillingen en extreme omgevingsomstandigheden. Het nadeel is dat sterke magnetische velden metingen van andere sensoren kunnen beïnvloeden. Er zijn ook drie grote groepen magnetische encoders.
					\begin{enumerate}
						\item \textbf{Magnetic Gear Tooth Sensor or Pickup}, Deze encoder heeft een magnetische sensor en een ferromagnetische tandwiel. Dit betekend dat het tandwiel in staat is om een magnetisch veld te geleiden. Het voordeel hiervan is dat deze goedkoop zijn, alleen zijn deze gelimiteerd door het aantal tanden. Dit limiteert de resolutie naar 120 of 240 PPR.
						\item \textbf{Magneto-Resistive}, Deze encoder genereert een sinus golf door weerstand te meten op een wiel met afwisselende magnetische polen of een film met weerstanden. De afwisselende magnetische polen bieden een hogere nauwkeurigheid dan het film met weerstanden. Deze encoders zijn moeilijker om te integreren in een systeem.
						\item \textbf{Hall-Effect magnetic}, Deze encoder maakt gebruik van het Hall-Effect. De encoder bevat een laag van een halfgeleider materiaal dat verbonden is met een voeding. als een magnetische pool langs de hall-effect sensor komt, word er een hoge voltage gegenereerd. De frequentie en amplitude van de verstoring in het magnetisch veld kan worden gebruikt om de snelheid en verplaatsing te bepalen. 
					\end{enumerate}


					\textbf{Optisch} Optische encoders zijn encoders die gebruik maken van optische signalen om stappen te tellen. Dit doen deze encoders door een schijf met patronen dat vast zit aan de as. Door een lichtsensor worden deze patronen omgezet in elektrische pulsen. Deze sensoren bieden hoge resolutie en nauwkeurigheid. Hierdoor zijn deze encoders goed op plekken waar veel nauwkeurigheid verwacht wordt. Het nadeel van deze sensoren is dat deze gevoeliger zijn voor stof, vuil en trillingen.

				\subsection{Encoder soorten}
					\textbf{Incrementeel}, Incrementele encoders genereren pulsen als de as beweegt. Op de schijf zijn er 1 of meerdere rijen met patronen. die rijen heten kanalen en als er twee of meer kanalen zijn, is het mogelijk om de richting bepalen. Deze patronen zijn 1/4 patroon van elkaar verschoven. Hierdoor krijg je bij een encoder met twee kanalen een patroon in de signalen. Dit betekend ook dat deze het zelfde aantal draden heeft als kanalen. Doordat de encoder een patroon volgt ten opzichte van een referentie punt, moet de microcontroller zelf de positie bijhouden. Hierdoor gaat de positiebepaling verloren als er een stroomonderbreking is.


					Het aantal patronen per kanaal bepaald de nauwkeurigheid. Indien een encoder meerdere kanalen heeft vergroot de nauwkeurigheid. bij 600 PPR (pulses per rotation)op één kanaal, heeft twee kanalen 600*4=2400 PPR. 
					

					Het patroon van de signalen is in graycode. Dit patroon is te zien in figuur x\todo{figuur nummer}. Graycode is een ander soort binaire code dat bij elke ophoging en verlaging één bit veranderd. De signalen moet je opvangen met een microcontroller via polling of interrupts. Interrupts zijn handiger omdat er dan geen data verloren kan gaan door een te kleine sample snelheid.
					\begin{figure}
						\begin{center}\begin{tabular}{ c|c }
						A & B \\
						\hline
						0 & 0 \\
						\hline
						1 & 0 \\
						\hline
						1 & 1 \\
						\hline
						0 & 1      
						\end{tabular}\end{center}
						\caption{Encoder sequentie}
					\end{figure}

					De signalen moeten worden opgevangen door een microcontroller. Dit kan via polling of interrupts. Interrupts stoppen het programma en voeren hun code uit, hierna kan het programma weer verder gaan. Polling is een manier van input lezen door iedere keer te kijken of de data veranderd is. Interrupts zijn hiervoor beter omdat er bij polling data verloren kan gaan door een te lage sample snelheid. De andere kant is dat een microcontroller maar een bepaald aantal interrupts per seconde kan verwerken.

					\textbf{Absolute}, Absolute encoders zijn encoders die elke positie op de schijf een unieke code geven. Hierdoor weet de encoder exact waar die zich bevind in de rotatie. Hierdoor gaat de positie niet verloren als de spanning onderbroken is. Absolute encoders onthouden niet standaard het aantal rotaties die zijn gedaan. Hiervoor zijn multi-turn varianten. Absolute encoders moeten worden uitgelezen met een protocol zoals SPI. Op een Absolute encoder bepaald het aantal bits hoe nauwkeurig de encoder is. Een multi-turn absolute encoder heeft x aantal bits per rotatie en x aantal bits voor de aantal rotaties. 
					
					Absolute encoders zijn duurder dan incrementele encoders. Van absolute zijn de multi-turn encoders duurder dan single-turn encoder.

					\begin{tabular}{|c|c|c|}
						\hline
						& absolute & Incrementeel\\
						\hline
						Kosten & Duur & goedkoper \\
						\hline
						uitlezen & protocol & pulsen tellen \\
						\hline
						positie onthouden zonder stroom & ja & nee \\
						\hline
					\end{tabular}

					{\color{white}-}\\
					Om zeker te weten of een incrementele encoder kan werken door middel van interrupts, is er een testje gedaan. Door de interrupt pin te verbinden met een andere pin die steeds hoog en laag gemaakt wordt, is er getest hoeveel interrupts de microcontroller kan krijgen in een seconde. Deze test geeft een idee hoeveel de microcontroller kan hebben. Hierin wordt de tijd gemeten door de tijd van de serial monitor.
					\begin{center}
						Op tijdstip 1: 09:36:47.772 waren er 848785 interrupts geweest. \\
						Op tijdstip 2: 09:36:53.204 waren er 1484065 interrupts geweest.\\ 

						Dit betekent dat er in 5.432 seconde 635280 interrupts zijn geweest. \\
						$5.432/635280 \approx 116951$ interrupts per seconde.
					\end{center}

					Een pulley van 16 tanden op een tandriem van 1mm tanden geeft 2mm op een tand. $2*16=32$mm per rotatie. Indien er 20000 interrupts per as gebruikt kunnen worden door de sensoren kan de freesbank in theorie $20000/2400*32\approx266.6$mm/s bewegen. Dit is genoeg dus deze soorten encoders zijn een optie.

					Na de opties voorgelegd te hebben aan de opdrachtgever heeft de opdrachtgever gekozen voor een betaalbare optische incrementele encoder. Dit is uiteindelijk de 'me38s6-c-(600)b5g2' rotary encoder geworden. Deze encoder heeft 600 PPR per kanaal. Dus 2400 PPR in dual channel. 
			\newpage
			\section{elektrische schema's}
			\begin{center}
			\includegraphics[scale=0.70]{CNC-1as-elektrisch-schema.pdf}\\
				Elektrisch schema één as
			\end{center}
			\begin{sidewaysfigure}
				\includegraphics[scale=0.60]{CNC-3asen-elektrisch-schema.pdf}
				\begin{center}
				Elektrisch schema drie assen
				\end{center}
			\end{sidewaysfigure}
	\newpage
	\Large
	\textbf{Changelog}
		\large\begin{center}
		\begin{tabular}{|c|c|}
			\hline
			Datum & verandering \\ 
			\hline
			1-12-2025 & Begonnen met stage verslag maken van losse documenten \\
			\hline
			10-12-2025 & Feedback verwerkt van studenten \\    
			\hline
			05-01-2026 & Document gereorganiseerd en analyse verbeterd \\
			\hline
			06-01-2026 & Stakeholders verplaatst, spelling, onderzoek verplaatst/herschreven \\
			\hline
			08-01-2026 & bijlage toegevoegd. \\
			\hline
		\end{tabular}
		\end{center}
\end{document}
__________________________________________________________________________________________________________________________________________________________________
----------------------------------------========================================[]========================================----------------------------------------
----------------------------------------========================================[]========================================----------------------------------------
----------------------------------------========================================[]========================================----------------------------------------
----------------------------------------========================================[]========================================----------------------------------------
----------------------------------------========================================[]========================================----------------------------------------
----------------------------------------========================================[]========================================----------------------------------------
----------------------------------------========================================[]========================================----------------------------------------
----------------------------------------========================================[]========================================----------------------------------------
----------------------------------------========================================[]========================================----------------------------------------
----------------------------------------========================================[]========================================----------------------------------------
----------------------------------------========================================[]========================================----------------------------------------
----------------------------------------========================================[]========================================----------------------------------------
----------------------------------------========================================[]========================================----------------------------------------
----------------------------------------========================================[]========================================----------------------------------------
----------------------------------------========================================[]========================================----------------------------------------
----------------------------------------========================================[]========================================----------------------------------------
----------------------------------------========================================[]========================================----------------------------------------
----------------------------------------========================================[]========================================----------------------------------------
----------------------------------------========================================[]========================================----------------------------------------
----------------------------------------========================================[]========================================----------------------------------------
----------------------------------------========================================[]========================================----------------------------------------
----------------------------------------========================================[]========================================----------------------------------------
----------------------------------------========================================[]========================================----------------------------------------
----------------------------------------========================================[]========================================----------------------------------------
----------------------------------------========================================[]========================================----------------------------------------
----------------------------------------========================================[]========================================----------------------------------------
----------------------------------------========================================[]========================================----------------------------------------
----------------------------------------========================================[]========================================----------------------------------------
----------------------------------------========================================[]========================================----------------------------------------
----------------------------------------========================================[]========================================----------------------------------------
----------------------------------------========================================[]========================================----------------------------------------
----------------------------------------========================================[]========================================----------------------------------------
----------------------------------------========================================[]========================================----------------------------------------
----------------------------------------========================================[]========================================----------------------------------------

[ ] Voorwoord
[x] Context waarin het stagebedrijf werkzaam is
[x] Stagebedrijf: Wat voor soort bedrijf is het? (werkveld, grootte, afdeling waar je stageloopt)
[x] Probleemstelling/ opdrachtbeschrijving/ eventueel een hoofdvraag en -deelvragen
[x] Stakeholders (+ op welke manier je ze betrekt bij jouw opdracht)
[ ] Scope en passende eisen die relevant zijn voor jouw opdracht
[x] Mogelijke oplossingen en je onderbouwde keuze(s)
[ ] Requirements
[ ] Bijbehorende ontwerpen
[ ] Realisatie (diverse iteraties/ prototypes), testen en testresultaten op basis van de requirements en
de ontwerpen.
[ ] Aanbevelingen
Bijlagen




het bedrijf + stakeholders
stakeholders: bedrijven -> pkmarine

analyse
wat is er nu? blokjes -> onderdelen
oude situatie -> nieuwe situatie
wat moet er komen?

snij kop

maten


				\begin{figure}
				\begin{center}\begin{tabular}{ c|c }
				A & B \\
				\hline
				0 & 0 \\
				\hline
				1 & 0 \\
				\hline
				1 & 1 \\
				\hline
				0 & 1      
				\end{tabular}\end{center}
				\caption{Encoder sequentie}
				\end{figure}

		peter.vanderklugt@pkmarine.nl

//Port ** Arduino Pin Number ** pin designation
PA 0 ** 22 ** D22	
PA 1 ** 23 ** D23	
PA 2 ** 24 ** D24	
PA 3 ** 25 ** D25	
PA 4 ** 26 ** D26	
PA 5 ** 27 ** D27	
PA 6 ** 28 ** D28	
PA 7 ** 29 ** D29	

PB 0 ** 53 ** SPI_SS	
PB 1 ** 52 ** SPI_SCK	
PB 2 ** 51 ** SPI_MOSI	
PB 3 ** 50 ** SPI_MISO	
PB 4 ** 10 ** PWM10	
PB 5 ** 11 ** PWM11	
PB 6 ** 12 ** PWM12	
PB 7 ** 13 ** PWM13	

PC 0 ** 37 ** D37	
PC 1 ** 36 ** D36	
PC 2 ** 35 ** D35	
PC 3 ** 34 ** D34	
PC 4 ** 33 ** D33	
PC 5 ** 32 ** D32	
PC 6 ** 31 ** D31	
PC 7 ** 30 ** D30	

PD 0 ** 21 ** I2C_SCL	
PD 1 ** 20 ** I2C_SDA	
PD 2 ** 19 ** USART1_RX	
PD 3 ** 18 ** USART1_TX	
PD 7 ** 38 ** D38	

PE 0 ** 0 ** USART0_RX	
PE 1 ** 1 ** USART0_TX	
PE 3 ** 5 ** PWM5	
PE 4 ** 2 ** PWM2	
PE 5 ** 3 ** PWM3	

PF 0 ** 54 ** A0	
PF 1 ** 55 ** A1	
PF 2 ** 56 ** A2	
PF 3 ** 57 ** A3	
PF 4 ** 58 ** A4	
PF 5 ** 59 ** A5	
PF 6 ** 60 ** A6	
PF 7 ** 61 ** A7	

PG 0 ** 41 ** D41	
PG 1 ** 40 ** D40	
PG 2 ** 39 ** D39	
PG 5 ** 4 ** PWM4	

PH 0 ** 17 ** USART2_RX	
PH 1 ** 16 ** USART2_TX	
PH 3 ** 6 ** PWM6	
PH 4 ** 7 ** PWM7	
PH 5 ** 8 ** PWM8	
PH 6 ** 9 ** PWM9	

PJ 0 ** 15 ** USART3_RX	
PJ 1 ** 14 ** USART3_TX	

PK 0 ** 62 ** A8	
PK 1 ** 63 ** A9	
PK 2 ** 64 ** A10	
PK 3 ** 65 ** A11	
PK 4 ** 66 ** A12	
PK 5 ** 67 ** A13	
PK 6 ** 68 ** A14	
PK 7 ** 69 ** A15	

PL 0 ** 49 ** D49	
PL 1 ** 48 ** D48	
PL 2 ** 47 ** D47	
PL 3 ** 46 ** D46	
PL 4 ** 45 ** D45	
PL 5 ** 44 ** D44	
PL 6 ** 43 ** D43	
PL 7 ** 42 ** D42	


encoders
x: PD0 en PD1 (20 en 21)
y: PD2 en PD3 (18 en 19)
z: PE4 en PE4 (2 en 3)

step pins
X: PA1 (23)
Y: PA3 (25)
Z: PA5 (27)

dir pins
X: PA0 (22)
Y: PA2 (24)
Z: PA4 (26)
pinouts:
https://devboards.info/boards/arduino-mega2560-rev3''

  /* model refrence control */

	todo: extra motor op elektrisch schema
