\documentclass{article}
\usepackage{graphicx} % Required for inserting images
\usepackage{todonotes}
\usepackage{hyperref}
\usepackage{listings}
\usepackage{rotating}
\usepackage{blindtext}
\usepackage[a4paper, total={6in, 8in}]{geometry}

\title{Stage Verslag 2025 - 2026}
\author{Olaf Goudriaan 1071349}
\date{TINSTG05 - Wessel Oele, Renee van Dorn}
\begin{document}
\sffamily
	\begin{titlepage}
		\begin{center}
			\includegraphics[width=8cm]{Sens2SeaLogo.png}

			\vspace*{4cm}
			\Huge
			\textbf{Aansturing Freesmachine} \\
			\LARGE
			\vspace{0.5cm}
        Stage verslag \\ 
				Sens2Sea
			
			\vspace{1.5cm}
					
			\huge
			\textbf{Olaf Goudriaan}
			\vspace{1.5cm}	
			\vfill
			\Large
			\begin{flushright}
				Geert Mosterdijk \\
				TINSTG05 \\
				Wessel Oele, Renee van Dorn \\
				1071349	\\
				2025 - 2026
			\end{flushright}
			
		\end{center}

	\end{titlepage}
	\newpage
 	\renewcommand{\familydefault}{\sfdefault}
	\tableofcontents
	\newpage
	\large

	\paragraph{Voorwoord}
	Dank gaat uit naar alle mensen die hebben meegeholpen aan deze stage. De medewerkers, begeleiders en andere mensen die hebben meegeholpen. Dit was een leerzame periode waarin veel geleerd is.  \todo{beter maken?  hoe???}

	\section{Context}
		\subsection{Het bedrijf}
			De stage heeft plaatsgevonden bij het bedrijf Sens2Sea. Sens2Sea is een klein bedrijf dat gevestigd is op het RDM terrein. Het bedrijf heeft een kavel in het innovation dock. Sens2Sea ontwikkelt radar technologieen voor sectoren zoals de scheepsvaart en andere maritieme sectoren. Het bedrijf wordt geleid door Geert Mosterdijk. Het bedrijf heeft maar een paar medewerkers. Deze zijn gespecialiseerd in hun vakgebieden zoals: elektronica en software. Verder lopen er bij Sens2Sea een aantal studenten stage van verschillende opleidingen. Het bedrijf werkt voor projecten samen met andere bedrijven. Deze bedrijven zijn tevens de stakeholders voor de projecten. Voor dit bedrijf geldt het volgende:
			\begin{enumerate}
				\item Sens2Sea. De opdracht is voor Sens2Sea, dit bedrijf bepaalt welke beslissingen gemaakt worden. Dit bedrijf heeft mensen in dienst die expertise bieden op gebied van elektronica en software. (Hoge invloed en Hoog belang)
				\item PKMarine. Dit bedrijf werkt samen met Sens2Sea en bied expertise in regeltechniek. (Hoge invloed, laag belang)
				\item CoE HRTech. Dit bedrijf financiert het project. (Lage invloed en hoog belang)
			\end{enumerate}

		\subsection{Opdracht}
			\paragraph{} Sens2Sea is bezig met radar technologie. Radar is een technologie die radiogolven gebruikt om, t.o.v. de positie van de radar, de afstand, snelheid en richting van objecten te bepalen. Radar kan gebruikt worden in goed en slecht weer. Hierdoor is het uitstekend geschikt om objecten te detecteren in sectoren zoals de scheepsvaart. Ook kan radar technologie gebruikt worden om de zeebodem in kaart te brengen en om olievlekken te detecteren. 

			\paragraph{} Sens2Sea is sinds kort bezig FMCW (Frequency Modulated Constant Wave) radar technologie. Hiervoor wilt Sens2Sea parabool antennes gebruiken. Deze antennes zorgen ervoor dat de radiogolven gebundeld worden. Door de bundeling van deze golven worden de signalen versterkt. Dit zorgt ervoor dat het beeld een betere resolutie heeft. Deze antennes zijn niet te verkrijgen in alle maten. Deze zouden custom op maat gemaakt moeten worden. Dit is ten opzichte van standaard geproduceerde maten een stuk duurder. Machines die al op de markt zijn en met een werkgrote van 4m x 1m x 0.4m, groot genoeg zijn kosten tienduizenden euro's. Daarp, wil Sens2Sea eigen antennes maken met een eigen 3D freesmachine. Deze opdracht omvat het realiseren van een nauwkeurige aansturing van een 3D freesmachine.
			
		\subsection{Scope} 
			\paragraph{} Om een freesmachine aan te sturen zijn er al opensource projecten zoals GRBL uitgevoerd. Deze projecten voldoen helaas niet aan de eisen omdat Sens2Sea wil dat de software en hardware makkelijk aanpasbaar is. Bovendien moet de freesmachine over een afstand van 4 meter nauwkeurig kunnen werken. Hiervoor is het van belang de aandrijving van de frees en de positie bepaling van de snijkop gescheiden te houden. Dit is niet het geval met deze opensource software zoals GRBL. Bij deze oplossingen bepaalt het systeem de positie van de kop door de stappen te tellen die de motoren moeten uitvoeren. Een probleem hiervan is dat de motoren een stap kunnen verspringen waardoor de gewenste qualiteit van de vorm van de antenne niet kan worden gegarandeerd. Daarom wil Sens2Sea dat de positie van de snijkop door sensoren wordt bepaald opdat deze kop nauwkeurig kan worden gepositioneerd in zowel de  als de Y en Z richtingen. 
			
			\paragraph{} De scope van het project omvat het scheiden van de aandrijving en de positie bepaling van de snijkop. Daarnaast is er sprake van aanpassen of herschrijven van GRBL om deze te laten werken met de sensoren. Ook de hardware aansturen/aansluiten hoort bij de scope.

		\subsection{Mogelijke oplossingen}
			Een mogelijke oplossing voor het probleem zouden stappenmotoren met encoder kunnen zijn. Dit zou handig zijn omdat de code van GRBL niet aangepast hoeft te worden. Deze encoders zitten op de achterkant van de encoder en meten of de encoder wel de stap heeft gezet die die moest zetten. Indien hier een fout in zit zal de encoder compenseren.

			Dit is voorgelegd aan de opdrachtgever. Maar dat is geen oplossing omdat afhankelijkheid van de fabrikant wordt gemaakt. Dit omdat de fabrikant de encoder bepaald op de motor en hierdoor de nauwkeurigheid van de encoder bepaald. Sens2Sea wilt een systeem dat aan te passen is naar de wensen die op dat moment belangrijk zijn. En indien nodig andere sensoren toevoegen/gebruiken voor de bepaling van de positie van de snijkop. 



	\newpage
	\section{Analyse}
		\subsection{Huidige situatie}
			\begin{figure}
				\begin{center}
				\includegraphics[width=12cm]{huidigsysteemdiagram2.png}
				\caption{systeem diagram}
				\end{center}
			\end{figure}
			De huidige situatie bestaat uit de onderdelen in figuur ... \todo{figuur}. In groen is aangegeven welke onderdelen al aanwezig zijn. Deze zijn: de motor drivers, motoren het frame en een frees. en in oranje is aangegeven welke onderdelen toegevoegd moeten worden.
			

			\todo{frees valt buiten scope, microstepping buiten scope, verwijzing datasheet?}
			\subsubsection{Huidige onderdelen}
				\paragraph{Frame}Het frame is gemaakt van hout. De werkruimte van het frame bedraagt \todo{groote van freesbank} x m in de lengte (X-as), xm in breedte (Y-as) en x m m in hoogte (Z-as). De totale buitenafmetingen van het frame zijn X m bij X m bij X m. langs beide zijden van de lengte-as zijn geleiderails gemonteerd. Hierover beweegt een wagentje. Op deze wagen is een tweede railsysteem bevestigd voor de beweging in de breedte-as.  Hierop bevind zich een een tweede wagentje voor de hoogte-as. De hoogte beweging wordt gerealiseerd door middel van een draaiende schroef, waardoor de snijkop gecontroleerd omhoog en omlaag kan bewegen. Aan deze snijkop is een frees gemonteerd. De aandrijving van de lengte- en breedte-as gebeurt met tandriemen. Deze lopen over de volledige lengte van de assen en worden aangedreven door timing pulleys. Dit zijn tandwielen voor tandriemen.

				\paragraph{Frees}Aan de hoogte-as is een frees bevestigd. Dit is een handfrees die op netstroom werkt. 

				\paragraph{Motoren}De motoren die gebruikt worden zijn NEMA 17 stappenmotoren. Deze motoren werken op 12-36V en mogen maximaal 2A hebben. Deze zitten bevestigd op het frame met brackets. Deze hebben een timing pulley waar de tandriem overheen loopt. De lengte as wordt aangedreven door 2 motoren. De breedte en hoogte assen worden aangedreven door 1 motor. 

				\paragraph{Drivers}De drivers die worden gebruikt voor de motoren zijn DRV8824 drivers. Deze drivers werken op 8.2-45V en mogen maximaal 2.5A hebben op 24V. Deze drivers maken gebruik van een STEP en DIR pin. Door de DIR pin hoog en laag te maken veranderd de polariteit over de aangesloten motor. Door de STEP pin hoog te maken zet de motor een stap. Deze drivers hebben ook de mogelijkheid om te microsteppen. Microsteppen maakt gebruik van sinusvormige stromen om meer nauwkeurigheid en hogere resolutie te krijgen. 

			\subsubsection{Nieuwe onderdelen}
			\todo{lab voeding gebruik ik, custom valt buiten scope}
				\paragraph{Voeding}De voeding is niet aanwezig voor het systeem. De microcontroller heeft 5V nodig en kan worden aangesloten op een laptop of powerbrick voor telefoons. De actuatoren hebben meer vermogen nodig en hiervoor word een labvoeding gebruikt

				\paragraph{Sensor}Uit een interview met de opdrachtgever is gebleken dat de opdrachtgever encoders wilt gebruiken en wilt bevestigen op een getande riem die op het frame is geplakt. Hierdoor kan de getande riem niet uitrekken. De nauwkeurigheid van de encoder is niet het belangrijkste omdat het project gaat over het scheiden van de aandrijving van de frees en positie bepaling van de snijkop. Verder doen de kosten er toe, maximaal 60 euro per encoder. Ook is er belang bij dat de de huidige positie opvragen snel kan aangezien een freesmachine een high performance machine is. Er zijn verschillende soorten encoders die verschillend werken. De nauwkeurigheid van een encoder wordt gemeten in Pulses Per Rotation(PPR). Dit geeft aan dat hoe hoger de PPR is de hoger de nauwkeurigheid omdat er meer pulses gegeven worden in de zelfde rotatie. Na het onderzoek zijn de opties voorgelegd aan de opdrachtgever en is er gekozen om een optische incrementele encoder te gebruiken.
				
				\paragraph{Microcontroller}Na de keuze voor een encoder is gekomen dat de microcontroller 6 interrupt pinnen nodig heeft. Verder moet deze rond de specificaties zijn als een arduino uno. Dit is omdat opensource oplossingen zoals GRBL gebaseerd zijn op een arduino uno. Hierom is voorgesteld om een arduino mega te gebruiken. Deze mircrocontroller heeft voldoende rekenkracht voor dit soort applicaties. Hiernaast heeft deze microcontroller ook de benodigde 6 interrupt pinnen voor de 3 verschillende assen. Daarnaast bieden de vele pinnen mogelijkheden om het systeem uit te breiden.

		\subsection{Requirements}
			Uit de analyse en aan de hand van meerdere interviews met de opdrachtgever zijn de volgende requirements opgekomen.
			\begin{center}
				\scalebox{0.711875}{\begin{tabular}{|c|c|c|c|}
					\hline
					Requirement & User story & Acceptatie & soort \\
					\hline
					De snijkop van de machine moet over & Als opdrachtgever wil ik dat de machine & Als de kop zich over de 	& Funct.	\\
					de X, Y en Z as kunnen bewegen. 		&	zich over de X, Y en Z as kan bewegen. 	& X, Y en Z as kan bewegen.	& 				\\
					\hline
					De positie van de snijkop moet  		& Als opdrachtgever wil ik dat de machine	& Als de de positie van de	& Funct. 	\\
					bepaald worden doormiddel van   		& de positie van de kop bepaald doormiddel& snijkop bepaald word door	&				 	\\
					sensoren.														& van sensoren on de nauwkeurigheid te 		& een sensor.								&				 	\\
																							& garanderen.															&														&					\\
					\hline
					Er kan g-code uitgevoerd worden			& Als opdrachtgever wil ik een g-code  		& Als de machine g-code kan & Funct. 	\\
					door de freesmachine. 							& bestand laten uitvoeren door de 				& uitvoeren.								&					\\
																							& freesmachine.														&														&					\\
					\hline
					De code van het prototype moet 			& Als opdrachtgever wil ik gemakkelijk 		& Als er in de code van de 	& Niet		\\
					modulair zijn waardoor gemakkelijk	& onderdelenkunnen vervangen en toevoegen & machine componenten				& Funct.	\\
					delen vervangen kunnen worden,			& aan de machine, zonder dat de hele code & toegevoegd, vervangen of	&					\\
				  zoals actuatoren en sensoren.				&	omgebouwd moet worden.									& verwijderd kunnen worden.	&					\\
					\hline
				\end{tabular} }
			\end{center}
	
	\section{Ontwerpen}
		\subsection{Tekeningen en diagrammen}
			Het elektrische schema voor een motor en encoder is te vinden in appendix B. Het schema bevat een arduino uno, DRV8824 stepper driver en een stepper motor. Het tweede schema bevat alle componenten voor 3 assen.
		\todo{mount encoder iteraties, }











	\section{Test Rapport}
		\todo{Work in progress}
		In dit plan worden de eisen/requirements van het prototype gestest om te kijken of ze behaald zijn.

		\subsection{De kop van de machine moet over drie assen bewegen}
			\subsubsection{Acceptatiecriteria:}
				Deze test is voltooid als de kop van de machine kan bewegen over de 3 bewegingsassen (x, y ,z).

			\subsubsection{Testopstelling}
				Deze test kan alleen op de machine worden uitgevoerd. Benodigd is: 
				\begin{enumerate}
					\item Testopstelling.
					\item Handleiding \todo{work in progress}
				\end{enumerate}

			\subsubsection{Instructies}
				\begin{enumerate}
					\item Volg de handleiding om de machine aan te zetten.
					\item Upload een simpel stuk gcode om te machine over alle 3 de assen te laten bewegen.
				\end{enumerate}

			\subsubsection{Verwacht resultaat}
				Het verwachte resultaat is dat de kop van de machine beweegt naar het punt uit de instructies.

			\subsubsection{Waarnemingen}
				\todo{waarnemingen invullen}
				

	\section{Aanbevelingen}
		\todo{aanbevelingen toeveogen}
	\section{Bronnen}
		\begin{enumerate}
			\item Wikipedia contributors, “Radar,” Wikipedia, The Free Encyclopedia (Dutch version). [Online]. Available: \url{https://nl.wikipedia.org/wiki/Radar}. Accessed: Jan. 6, 2026.
			\item Wikipedia contributors, “Radar,” Wikipedia, The Free Encyclopedia. [Online]. Available: \url{https://en.wikipedia.org/wiki/Radar}. Accessed: Jan. 6, 2026.
			\item Wikipedia contributors, “Incremental encoder,” Wikipedia, The Free Encyclopedia. [Online]. Available: \url{https://en.wikipedia.org/wiki/Incremental_encoder}. Accessed: Jan. 6, 2026.
			\item Wikipedia contributors, “Rotary encoder,” Wikipedia, The Free Encyclopedia. [Online]. Available: \url{https://en.wikipedia.org/wiki/Rotary_encoder}. Accessed: Jan. 6, 2026.
			\item Laumans Techniek, “Encoders voor industriële automatisering,” Laumans Techniek. [Online]. Available: \url{https://www.laumanstechniek.nl/blog/kennisbank-3/encoders-voor-industriele-automatisering-20}. Accessed: Jan. 6, 2026.
			\item PBC Linear, “Stepper Motor Support: NEMA 17 Data Sheet,” PBC Linear. [Online]. Available: \url{https://media.pbclinear.com/pdfs/pbc-linear-data-sheets/data-sheet-stepper-motor-support.pdf}. Accessed: Jan. 6, 2026.
			\item Texas Instruments, “DRV8824 Stepper Motor Controller IC,” Datasheet. [Online]. Available: \url{https://www.ti.com/lit/ds/symlink/drv8824.pdf}. Accessed: Jan. 6, 2026.
			\item Linear Motion Tips, “Microstepping Basics,” Linear Motion Tips. [Online]. Available: \url{https://www.linearmotiontips.com/microstepping-basics/}. Accessed: Jan. 6, 2026.
			\item Dynapar, “Magnetic Encoders,” Dynapar Knowledge Center. [Online]. Available: \url{https://www.dynapar.com/knowledge/encoder-basics/encoder-technology/magnetic-encoders}. Accessed: Jan. 8, 2026.
			\item Wikipedia contributors, “Gray code,” Wikipedia, The Free Encyclopedia. [Online]. Available: \url{https://en.wikipedia.org/wiki/Gray_code}. Accessed: Jan. 8, 2026. 
		\end{enumerate}


		\appendix
		
			\section{Encoder onderzoek}
				Voor de positie van de snijkop van een freesbank te bepalen zijn verschillende mogelijkheden. Uit een interview met de opdrachtgever is gekomen dat deze encoders wilt gebruiken. Een encoder is een sensor met een roterende as. Als deze as draait zet de encoder de beweging om in elektrische signalen. Hierdoor kunnen microcontrollers de positie, snelheid en richting bepalen. De encoders zijn onderteverdelen in verschillende soorten: Incrementeel of absoluut. Ook zijn er verschillende meet methodes: Magnetisch of optisch. Er zijn ook encoders die linear zijn, maar daarvan heeft de opdrahctgever aangegeven dat die niet bruikbaar zijn. Bij de keuze moet ook rekening gehouden worden met dat de prijs van een sensor onder de 60 euro moet blijven. 

				\subsection{Meet methodes}
					Er zijn verschillende manieren om de beweging te meten. Hiervoor zijn optisch en magnetisch de meest gebruikte technologieen.
					\textbf{Magnetisch} Magnetische encoders maken gebruik van een magnetisch veld om de rotatie van de as te bepalen. Dit heeft het voordeel dat deze robuuster zijn. Hiernaast zijn deze ook beter bestand tegen vervuiling, trillingen en extreme omgevingsomstandigheden. Het nadeel is dat sterke magnetische velden metingen van andere sensoren kunnen beinvloeden. Er zijn ook drie grote groepen magnetische encoders.
					\begin{enumerate}
						\item \textbf{Magnetic Gear Tooth Sensor or Pickup}, Deze encoder heeft een magnetiche sensor en een ferromagnetische tandwiel. Dit betekend dat het tandwiel in staat is om een magnetisch veld te geleiden. Het voordeel hiervan is dat deze goedkoop zijn, alleen zijn deze gelimiteerd door het aantal tanden. Dit limiteerd de resolutie naar 120 of 240 PPR.
						\item \textbf{Magneto-Resistive}, Deze encoder genereert een sinus golf door weerstand te meten op een wiel met afwisselende magnetische polen of een film met weerstanden. De afwisselende magnetische polen bieden een hogere nauwkeurigheid dan het film met weerstanden. Deze encoders zijn moeilijker om te intergreren in een systeem.
						\item \textbf{Hall-Effect magnetic}, Deze encoder maakt gebruik van het Hall-Effect. De encoder bevat een laag van een halfgeleider materiaal dat verbonden is met een voeding. als een magnetische pool langs de hall-effect sensor komt, word er een hoge voltage gegenereerd. De frequentie en amplitude van de verstoring in het magnetisch veld kan worden gebruikt om de snelheid en verplaatsing te bepalen. 
					\end{enumerate}


					\textbf{Optisch} Optische encoders zijn encoders die gebruik maken van optische signalen om stappen te tellen. Dit doen deze encoders door een schijf met patronen dat vast zit aan de as. Door een lichtsensor worden deze patronen omgezet in elektrische pulsen. Deze sensoren bieden hoge resolutie en nauwkeurigheid. Hierdoor zijn deze encoders goed op plekken waar veel nauwkeurigheid verwacht wordt. Het nadeel van deze sensoren is dat deze gevoeliger zijn voor stof, vuil en trillingen.

				\subsection{Encoder soorten}
					\textbf{Incrementeel}, Incrementele encoders genereren pulsen als de as beweegt. Op de schijf zijn er 1 of meerdere rijen met patronen. die rijen heten kanalen en als er 2 of meer kanalen zijn, is het mogelijk om de richting bepalen. Deze patronen zijn 1/4 patroon van elkaar verschoven. Hierdoor kijg je bij een encoder met 2 kanalen een patroon in de signalen. Dit betekend ook dat deze het zelfde aantal draden heeft als kanalen. Doordat de encoder een patroon volgt ten opzichte van een referentie punt, moet de microcontroller zelf de positie bijhouden. Hierdoor gaat de positiebepaling verloren als er een stroomonderbreking is.

					Het aantal patronen per kanaal bepaald de nauwkeurigheid. Indien eene encoder meerdere kanalen heeft vergroot de nauwkeurigheid. bij 600 PPR (pulses per rotation)op 1 kanaal, heeft 2 kanalen 600*4=2400 PPR. 
					
					Het patroon van de signalen is in graycode. Dit patroon is te zien in figuur x\todo{figuur nummer}. Graycode is een ander soort binare code dat bij elke increment en decrement 1 bit veranderd. De signalen moet je opvangen met een microcontroller via polling of interrupts. Interrupts zijn handiger omdat er dan geen data verloren kan gaan door een te kleine sample snelheid.
					\begin{figure}
						\begin{center}\begin{tabular}{ c|c }
						A & B \\
						\hline
						0 & 0 \\
						\hline
						1 & 0 \\
						\hline
						1 & 1 \\
						\hline
						0 & 1      
						\end{tabular}\end{center}
						\caption{Encoder sequentie}
					\end{figure}

					De signalen moeten worden opgevangen door een microcontroller. Dit kan via polling of interrupts. Interrupts stoppen het programma en voeren hun code uit, hierna kan het programma weer verder gaan. Polling is een manier van input lezen door iedere keer te kijken of de data veranderd is. Interrupts zijn hiervoor beter omdat er bij polling data verloren kan gaan door een te lage sample snelheid. De andere kant is dat een microcontroller maar een bepaald aantal interrupts per seconde kan verwerken.

					\textbf{Absolute}, Absolute encoders zijn encoders die elke positie op de schijf een unieke code geven. Hierdoor weet de encoder exact waar die zich bevind in de rotatie. Hierdoor gaat de positie niet verloren als de spanning onderbroken is. Absolute encoders onthouden niet standaard het aantal rotaties die zijn gedaan. Hiervoor zijn multi-turn varianten. Absolute encoders moeten worden uitgelezen met een protocol zoals SPI. Op een Absolute encoder bepaald het aantal bits hoe nauwkeurig de encoder is. Een multi-turn absolute encoder heeft x aantal bits per rotatie en x aantal bits voor de aantal rotaties. 
					
					Absolute encoders zijn duurder dan incrementele encoders. Van absolute zijn de multi-turn encoders duurder dan single-turn encoder.

					\begin{tabular}{|c|c|c|}
						& absolute & Incrementeel\\
						Kosten & Duur & goedkoper \\
						uitlezen & protocol & pulsen tellen \\
						positie onthouden zonder stroom & ja & nee \\
					\end{tabular}

					Na de opties voorgelegd te hebben aan de opdrachtgever heeft de opdrachtgever gekozen voor een optische incrementele encoder. 
			\newpage
			\section{elektrische schemas}
			\begin{center}
			\includegraphics[scale=0.70]{CNC-1as-elektrisch-schema.pdf}\\
				Elektrisch schema 1 as
			\end{center}
			\begin{sidewaysfigure}
				\includegraphics[scale=0.70]{CNC-3asen-elektrisch-schema.pdf}
				\begin{center}
				Elektrisch schema 3 asen
				\end{center}
			\end{sidewaysfigure}
	\newpage
	\Large
	\textbf{Changelog}
		\large\begin{center}
		\begin{tabular}{|c|c|}
			\hline
			Datum & verandering \\ 
			\hline
			1-12-2025 & Begonnen met stage verslag maken van losse documenten \\
			\hline
			10-12-2025 & Feedback verwerkt van studenten \\    
			\hline
			05-01-2026 & Document herordend en analyse verbeterd \\
			\hline
			06-01-2026 & Stakeholders verplaatst, spelling, onderzoek verplaatst/herschreven \\
			\hline
			08-01-2026 & bijlage toegevoegd. \\
			\hline
		\end{tabular}
		\end{center}
\end{document}
__________________________________________________________________________________________________________________________________________________________________
----------------------------------------========================================[]========================================----------------------------------------
----------------------------------------========================================[]========================================----------------------------------------
----------------------------------------========================================[]========================================----------------------------------------
----------------------------------------========================================[]========================================----------------------------------------
----------------------------------------========================================[]========================================----------------------------------------
----------------------------------------========================================[]========================================----------------------------------------
----------------------------------------========================================[]========================================----------------------------------------
----------------------------------------========================================[]========================================----------------------------------------
----------------------------------------========================================[]========================================----------------------------------------
----------------------------------------========================================[]========================================----------------------------------------
----------------------------------------========================================[]========================================----------------------------------------
----------------------------------------========================================[]========================================----------------------------------------
----------------------------------------========================================[]========================================----------------------------------------
----------------------------------------========================================[]========================================----------------------------------------
----------------------------------------========================================[]========================================----------------------------------------
----------------------------------------========================================[]========================================----------------------------------------
----------------------------------------========================================[]========================================----------------------------------------
----------------------------------------========================================[]========================================----------------------------------------
----------------------------------------========================================[]========================================----------------------------------------
----------------------------------------========================================[]========================================----------------------------------------
----------------------------------------========================================[]========================================----------------------------------------
----------------------------------------========================================[]========================================----------------------------------------
----------------------------------------========================================[]========================================----------------------------------------
----------------------------------------========================================[]========================================----------------------------------------
----------------------------------------========================================[]========================================----------------------------------------
----------------------------------------========================================[]========================================----------------------------------------
----------------------------------------========================================[]========================================----------------------------------------
----------------------------------------========================================[]========================================----------------------------------------
----------------------------------------========================================[]========================================----------------------------------------
----------------------------------------========================================[]========================================----------------------------------------
----------------------------------------========================================[]========================================----------------------------------------
----------------------------------------========================================[]========================================----------------------------------------
----------------------------------------========================================[]========================================----------------------------------------
----------------------------------------========================================[]========================================----------------------------------------

[ ] Voorwoord
[x] Context waarin het stagebedrijf werkzaam is
[x] Stagebedrijf: Wat voor soort bedrijf is het? (werkveld, grootte, afdeling waar je stageloopt)
[x] Probleemstelling/ opdrachtbeschrijving/ eventueel een hoofdvraag en -deelvragen
[x] Stakeholders (+ op welke manier je ze betrekt bij jouw opdracht)
[ ] Scope en passende eisen die relevant zijn voor jouw opdracht
[x] Mogelijke oplossingen en je onderbouwde keuze(s)
[ ] Requirements
[ ] Bijbehorende ontwerpen
[ ] Realisatie (diverse iteraties/ prototypes), testen en testresultaten op basis van de requirements en
de ontwerpen.
[ ] Aanbevelingen
Bijlagen




het bedrijf + stakeholders
stakeholders: bedrijven -> pkmarine

analyse
wat is er nu? blokjes -> onderdelen
oude situatie -> nieuwe situatie
wat moet er komen?

snijkop

maten

				\begin{center}
					Op tijdstip 1: 09:36:47.772 waren er 848785 interrupts geweest. \\
					Op tijdstip 2: 09:36:53.204 waren er 1484065 interrupts geweest.\\ 

					Dit betekent dat er in 5.432 seconde 635280 interrupts zijn geweest. \\
					$5.432/635280 \approx 116951$ interrupts per seconde.
				\end{center}

				\begin{figure}
				\begin{center}\begin{tabular}{ c|c }
				A & B \\
				\hline
				0 & 0 \\
				\hline
				1 & 0 \\
				\hline
				1 & 1 \\
				\hline
				0 & 1      
				\end{tabular}\end{center}
				\caption{Encoder sequentie}
				\end{figure}

		peter.vanderklugt@pkmarine.nl

//Port ** Arduino Pin Number ** pin designation
PA 0 ** 22 ** D22	
PA 1 ** 23 ** D23	
PA 2 ** 24 ** D24	
PA 3 ** 25 ** D25	
PA 4 ** 26 ** D26	
PA 5 ** 27 ** D27	
PA 6 ** 28 ** D28	
PA 7 ** 29 ** D29	

PB 0 ** 53 ** SPI_SS	
PB 1 ** 52 ** SPI_SCK	
PB 2 ** 51 ** SPI_MOSI	
PB 3 ** 50 ** SPI_MISO	
PB 4 ** 10 ** PWM10	
PB 5 ** 11 ** PWM11	
PB 6 ** 12 ** PWM12	
PB 7 ** 13 ** PWM13	

PC 0 ** 37 ** D37	
PC 1 ** 36 ** D36	
PC 2 ** 35 ** D35	
PC 3 ** 34 ** D34	
PC 4 ** 33 ** D33	
PC 5 ** 32 ** D32	
PC 6 ** 31 ** D31	
PC 7 ** 30 ** D30	

PD 0 ** 21 ** I2C_SCL	
PD 1 ** 20 ** I2C_SDA	
PD 2 ** 19 ** USART1_RX	
PD 3 ** 18 ** USART1_TX	
PD 7 ** 38 ** D38	

PE 0 ** 0 ** USART0_RX	
PE 1 ** 1 ** USART0_TX	
PE 3 ** 5 ** PWM5	
PE 4 ** 2 ** PWM2	
PE 5 ** 3 ** PWM3	

PF 0 ** 54 ** A0	
PF 1 ** 55 ** A1	
PF 2 ** 56 ** A2	
PF 3 ** 57 ** A3	
PF 4 ** 58 ** A4	
PF 5 ** 59 ** A5	
PF 6 ** 60 ** A6	
PF 7 ** 61 ** A7	

PG 0 ** 41 ** D41	
PG 1 ** 40 ** D40	
PG 2 ** 39 ** D39	
PG 5 ** 4 ** PWM4	

PH 0 ** 17 ** USART2_RX	
PH 1 ** 16 ** USART2_TX	
PH 3 ** 6 ** PWM6	
PH 4 ** 7 ** PWM7	
PH 5 ** 8 ** PWM8	
PH 6 ** 9 ** PWM9	

PJ 0 ** 15 ** USART3_RX	
PJ 1 ** 14 ** USART3_TX	

PK 0 ** 62 ** A8	
PK 1 ** 63 ** A9	
PK 2 ** 64 ** A10	
PK 3 ** 65 ** A11	
PK 4 ** 66 ** A12	
PK 5 ** 67 ** A13	
PK 6 ** 68 ** A14	
PK 7 ** 69 ** A15	

PL 0 ** 49 ** D49	
PL 1 ** 48 ** D48	
PL 2 ** 47 ** D47	
PL 3 ** 46 ** D46	
PL 4 ** 45 ** D45	
PL 5 ** 44 ** D44	
PL 6 ** 43 ** D43	
PL 7 ** 42 ** D42	


encoders
x: PD0 en PD1 (20 en 21)
y: PD2 en PD3 (18 en 19)
z: PE4 en PE4 (2 en 3)

step pins
X: PA1 (23)
Y: PA3 (25)
Z: PA5 (27)

dir pins
X: PA0 (22)
Y: PA2 (24)
Z: PA4 (26)
pinouts:
https://devboards.info/boards/arduino-mega2560-rev3